%!TEX program = xelatex
\documentclass{beamer}

\def\code#1{\texttt{#1}}

\def\image[#1][#2]{
	\begin{figure}[H]
		\centering
		\includegraphics[#2]{#1}
\end{figure}}

\def\captionimage[#1][#2]#3{
	\begin{figure}[H]
		\centering
		\includegraphics[#2]{#1}
		\caption{#3}
\end{figure}}

\logo{\includegraphics[height=1cm]{logo.png}\vspace{239pt}}


\usetheme{mhthm}

\title{GeoNews}
\subtitle{Università degli Studi di Perugia}
\author{Giorgio Mazza}
\institute{UniPG}


\date{Luglio 2018}

\setcounter{showSlideNumbers}{1}

\begin{document}
	\setcounter{showProgressBar}{0}
	\setcounter{showSlideNumbers}{0}

	\frame{\titlepage}
	\begin{frame}
		\frametitle{Sommario}
		\begin{itemize}
			\item Introduzione \\ {\footnotesize\hspace{1em} Caratteristiche generali}
			\item Client \\ {\footnotesize\hspace{1em} Aspetti relativi all'applicazione come client}
			\item Server \\ {\footnotesize\hspace{1em} Aspetti riguardanti il server personale dei commenti}
		\end{itemize}
	\end{frame}

	\setcounter{framenumber}{0}
	\setcounter{showProgressBar}{1}
	\setcounter{showSlideNumbers}{1}
	\section{Introduzione}
		\begin{frame}
			\frametitle{Caratteristiche generali}
			\begin{itemize}
				\item Applicazione Android in grado di ricercare notizie e articoli da
oltre 30.000 fonti
				\item Notizie visualizzate su una mappa Google (API) posizionate contestualmente

				
			\end{itemize}
		\end{frame}
		\begin{frame}
					\begin{itemize}
			\frametitle{Linguaggi}
				\item Linguaggio client: \textbf{Kotlin}
				
				\item Linguaggio scripting server: \textbf{PHP}
				\item Linguaggio Database online: \textbf{MySQL}
				\item Uso di \textbf{API REST} di newsapi.org
				\item Scambio oggetti con \textbf{JSON}
								
			\end{itemize}
		\end{frame}
		\begin{frame}
			\frametitle{Librerie}
			\begin{itemize}
				\item \textbf{OkHTTP} \\ {\footnotesize\hspace{1em} client HTTP semplice ed efficiente}
				\item \textbf{Picasso} \\ {\footnotesize\hspace{1em} per il caricamento e caching delle immagini} 
				\item \textbf{Gson} \\ {\footnotesize\hspace{1em} per il parsing JSON}
				
			\end{itemize}
		\end{frame}
		
				\begin{frame}
			\frametitle{Architettura}
			\begin{itemize}
				\item Architettura dell'applicazione
				\image[images/Diagram.png][scale=0.3]
			\end{itemize}
		\end{frame}

	
	\section{Client}
		\begin{frame}
			\frametitle{Fetch Articles}
			\begin{itemize}
			\item Operazioni di reperimento degli articoli svolte dal file FetchArticles.kt, che effettua una chiamata \textbf{GET} in forma:\\ 
\textbf{http://newsapi.org/v2/\textbf{end-point} ? \textbf{user-queries} \& apiKey=priv-key} 
\item Stringa \textbf{JSON} passata alla funzione gson di Gson, che effettuerà il parsing
			\end{itemize}
			

		\end{frame}
		
			\begin{frame}
			\frametitle{Fetch Comments}
			 Nel fragment dei commenti (\code{ArticleCommentFragment}), sono chiamate le funzioni CRUD sul database nel server di \code{geonews.altervista.org}, mediante l'uso di \code{okHttp}:
\begin{itemize}
\item\textbf{CREATE}: \\ \code{createComment()} invia una richiesta \textbf{POST} \\code{http://geonews.altervista.org/addComment.php} \\attraverso la pressione di un bottone "invia"
\item\textbf{READ}: \code{fetchComments()} invia una richiesta \textbf{GET} \code{http://geonews.altervista.org/getAllComments.php} \\ effettuando uno \textit{swipe to refresh}
\item\textbf{UPDATE}: \code{updateComment()} invia una richiesta \textbf{POST} \code{http://geonews.altervista.org/updateComment.php} \\dopo che l'utente ha scelto \textit{aggiorna commento}
\item\textbf{DELETE}: \code{deleteComment()} invia una richiesta \textbf{POST} \code{http://geonews.altervista.org/deleteComment.php} \\dopo che l'utente ha scelto \textit{cancella commento}
\end{itemize}
\end{frame}

	
	\section{Server}
		\begin{frame}	
			\frametitle{Database}
			Lo schema della tabella 
			\textbf{COMMENTS} è il seguente:
\image[images/DB_structure.png][scale=0.6]

		\end{frame}		
		\begin{frame}
			\frametitle{Script PHP - CRUD}
			\begin{itemize}
\item\textbf{addComment.php} operazioni di \code{INSERT} di un commento mediante una richiesta \code{GET}
\item\textbf{getAllComment.php} per le operazioni di \code{SELECT} su tutti i commenti di un determinato articolo, questo grazie alla clausola\\ \code{WHERE url=\$url\_articolo}
\item\textbf{updateComment.php} per le operazioi di \code{UPDATE} di un commento. Con la clausola\\ \code{WHERE id=\$id\_commento} si prende il commento da modificare.
\item\textbf{deleteComment.php} per le operazioi di \code{DELETE} di un commento. 
\end{itemize}
		\end{frame}	
		
				\begin{frame}
			\frametitle{Script PHP - prepared statement}
			\begin{itemize}
\item\textbf{addComment.php} 
\image[images/prepared.png][scale=0.5]
\end{itemize}
		\end{frame}	

		
	
\end{document}
